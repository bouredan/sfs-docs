\chap Implementace
Výsledek této práce je prototyp sémantického facetového vyhledávače na platformě React. 
Tento prototyp je popsán v této kapitole, společně s návrhem jeho architektury a zdůvodněním některých rozhodnutí při vývoji. 
Prvně však zmíníme použité technologie a knihovny.

\sec Technologie a knihovny
Jako součást zadání této práce bylo rozhodnuto, že na implementaci prototypu bude využit javascriptový framework React. 
Ten je v současné době velmi populární a pro naše potřeby vhodný, díky jeho modularizaci. 
Použili jsme tedy doporučovanou metodu create-react-app k vytvoření základních modulů. 
Abychom se nemuseli zaobírat UI prvky prototypu, využili jsme knihovnu MUI (dříve známá také jako Material-UI) a použili připravené komponenty z ní.
Druhá knihovna, kterou využíváme je fetch-sparql-endpoint \cite[fetch-sparql-endpoint]. 
Ta nám zjednodušuje volání \glref{SPARQL} endpointu a serializaci příchozích dat do formátu RDFJS. 
Vybrali jsme ji hlavně proto, že je skutečně jednoduchá, neimplementuje nepotřebné věci navíc a je udržována aktuální.
Všechen kód je napsaný v Typescriptu, což je nadstavba Javascriptu, která jej rozšiřuje o statické typování a předchází tak spoustě chyb.

\sec Architektura
Při návrhu architektury je nutné dbát na to, aby byl modul vyhledávací logiky nezávislý na použitou platformu. 
Toho lze docílit buďto jeho realizací jako samostatné serverové služby nebo jako modul bez využití konstruktů využité platformy. 
Rozdělili jsme tedy vyhledávač na modul vyhledávací logiky napsaný v čistém Typescriptu se jménem sfs-api 
a modul napojení na prezentační vrstvu (web elementy) určený pro React nazvaný sfs-ui. 
Prototyp pak oba tyto moduly importuje, aby je ukázal na příkladu.

Pro použití modulu sfs-api musíme vytvořit instanci typu FacetSearchApi, kde předáváme konfiguraci požadovaných facetů a \glref{SPARQL} endpointu.
\begtt 
export const sfsApi = new FacetSearchApi({
  endpointUrl: "https://dbpedia.org/sparql",
  facetConfigs: facetConfigs,
  prefixes: `
        PREFIX rdfs: <http://www.w3.org/2000/01/rdf-schema#>
        PREFIX dbp: <http://dbpedia.org/property/>
        PREFIX dbo: <http://dbpedia.org/ontology/>
        PREFIX foaf: <http://xmlns.com/foaf/0.1/>
      `
});
\endtt

Facet je reprezentován rozhraním Facet a má své unikátní facetId, dle kterého je dále identifikován.
Jednotlivé typy facetů pak toto rozhraní rozšiřují a implementují jeho metodu \uv{generateSparql}, která se volá při sestavení výsledného \glref{SPARQL} dotazu.
Vracet by měla část \glref{SPARQL} týkající se tohoto facetu.
Podpora dalších typů je plánovaná s dalším vývojem vyhledávače. 
Tento design umožňuje uživateli vytvořit si vlastní typ facetu (implementováním interfacu Facet) pro případy nepokryté knihovnou, s tím, že ho ale stále může kombinovat s ostatními facety.

K připojení jednotlivých facetů používáme modul sfs-ui. 
Ten za pomoci implementovaného React hooku spojí prezentační vrstvu se stavem facetu a je tak bodem komunikace s druhým modulem.
K tomu se používá hlavně subscriber pattern, kde je identifikátorem právě ono facetId.
