\chap Facetové vyhledávání

\sec Popis
Z hlediska SEO je důležité, aby nevznikali duplikátní URL pro stejný obsah. \cite[facet-search-best-practices].
    
\sec Typy facetů
Typy facetů lze řadit dle toho jaké hodnoty a jakým způsobem jdou navolit. 
    \secc Basic facet
    Základní typ facetu s možností volby nejvýše jedné hodnoty podle které je pak sada výsledků filtrována. Ovládácím prvkem bývá select element.

    \secc Range facet
	Facet s možností nastavení rozsahu

	\secc Bucket facet
	Podobné jako basic facet, akorát kategorie jsou vytvořeny dle částí rozsahu.
		
\sec Srovnání přístupů
Řešení jak implementovat facetové vyhledávání je více. Lze je rozdělit do 2 hlavních kategorií podle toho, kde dochází k filtraci výsledků dle facetů, tedy buďto na 
klientské straně či na straně serveru. Mezi těmito kategoriemi se pak rozhodujeme podle toho, co je v našem případě rychlejší, častější se však zdá být filtrování na straně serveru.
    \secc Staticky definované facety
	1. fetchovani jen results co chci (filter na serveru) vs filter na client side
	3. search results obsahují facety eg. https://www.npmjs.com/package/@ebi-gene-expression-group/scxa-faceted-search-results
	
    \sec Převedení do sémantického světa
	.
