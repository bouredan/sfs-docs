\chap Facetové vyhledávání
V této kapitole si popíšeme co je facetové vyhledávání a k čemu se primárně využívá. Zanalyzujeme a srovnáme pak různé přístupy k implementaci
facetového vyhledávání, především z hlediska využití sémantických technologií. Abychom získali přehled o používaných řešení facetového vyhledávání, zanalyzujeme 
poskytované Facet Search APIs největších společností v této oblasti jako Elastic či Solr.

\sec Popis
\medskip
\picw=15cm \cinspic images/facets-explained.jpeg
\caption/f Ukázka facetů s vysvětlivkami. \cite[facets-explained-image]
\medskip
Facetové vyhledávání je zatřídění vyhledaných výsledků do různých kategorií (facetů) dle kterých se dá sada výsledků dále filtrovat.
Dá se ním tak obohatit každé vyhledávání, ale často bývá spojeno s fulltextovým vyhledáváním (TODO možná někde vysvětlit co to je), aby uživateli umožnilo jeho dotaz dále upřesnit.
Hojně se využívá třeba v e-commerce sektoru, kde podle studie Nielsen Norman Group (NNG) z roku 2018 jsou e-shopy bez facetového vyhledávání výjimkou.\cite[nng-study]
Jelikož není definovaný žádný standard facetového vyhledávání, zadefinujeme si, co by měl takový modul facetového vyhledávání splňovat:
\begitems
* facet obsahuje hodnotu pro každý výsledek ze sady výsledků
* jednotlivé facety lze kombinovat mezi sebou
* mezi kritérii facetů platí logický AND (ne pouze OR), tzn. aby se výsledek objevil v sadě výsledku, musí vyhovět všem aktivním facetům
* hodnoty facetů ukazují počet výsledků, které aplikování facetu s danou hodnotou v aktuálním stavu vrátí
* hodnoty facetů, které by vrátily prázdnou sadu výsledků se nezobrazují nebo jsou \uv{disabled}\fnote{Jejich HTML ovládací prvek má atribut disabled.}
\enditems

\sec Typy facetů
Typy facetů řadíme dle toho jaké hodnoty a jakým způsobem jdou navolit. 
    \secc Basic facet
    Základní typ facetu s možností volby nejvýše jedné hodnoty podle které je pak sada výsledků filtrována. Ovládácím prvkem bývá select element.

    \secc Checkbox facet
    Facet s možností volby více hodnot skrz zaškrtávání checkboxů.

    \secc Range facet
	Facet pro číslená data s možností nastavení rozsahu. Ovládácím prvkem bývá posuvník (input element s hodnoutou atributu type range).

	\secc Bucket facet
	Podobné jako range facet, akorát se neovládá posuvníkem, ale jsou nadefinovány rozsahy, do kterých se pak výsledky roztřídí.
		
\sec Srovnání přístupů
Řešení jak implementovat facetové vyhledávání je více. 
Obecně je lze rozdělit podle toho, kde dochází k filtraci výsledků aktivními facety, tedy buďto na straně klienta či na straně serveru.
U implementace facetového vyhledávání je také nutné myslet na to jakým způsobem se budou plnit data facetů. Nejčastěji se setkáváme s tím, že se
hodnoty facetů posílají ve stejné response jako sada výsledků. TODO modul vs server ještě

    \secc Filtrování na straně klienta

    \secc Filtrování na straně serveru
	search results obsahují facety eg. https://www.npmjs.com/package/@ebi-gene-expression-group/scxa-faceted-search-results
	
    \sec Převedení do sémantického světa
	.