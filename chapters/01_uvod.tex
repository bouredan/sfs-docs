\chap Úvod
Málokterý vynález ovlivnil svět v takové míře jako vznik World Wide Web (zkráceně WWW či web). Za poměrně krátkou dobu své existence se web rozšířil téměř do každé části našeho života a dnes si bez něj 
lze svět jen těžko představit. Oproti ostatním ICT technologiím, které se často výrazně inovují a mění každých několik let, web funguje už 20 let téměř stejně.
To se však začíná měnit s příchodem Sémantického webu, který zásadně ovlivňuje, jak přistupujeme k datům v internetu – místo relací mezi dokumenty (hypertextové odkazy) můžeme vytvářet relace mezi fakty. 
Svět lze tak mnohem lépe popsat a stává se pro nás srozumitelnější. Navíc jsou tyto relace více strojově čitelné, tudíž můžeme nad těmito daty mnohem přesněji vyhledávat informace či vykonávat automatizace. 


Tato změna si žádá nové přístupy k ukládání, zpracování a vyhledávání dat. Právě vyhledáváním v sémantických datech se bude zabývat tato práce, konkrétně facetovým vyhledáváním. Facetové vyhledávání, tedy 
zatřídění entit do různých kategorií, je v současné době velmi rozšířené. Pomáhá nám upřesnit výsledky vyhledávání a najdeme jej tedy skoro v každém větším e-shopu. Přístupů k facetovému vyhledávání je více, ne 
všechny jsou však vhodné pro sémantická data. Nad sémantickými daty tak existuje velmi málo řešení facetového vyhledávání a k tomu jsou často závislé na nějaké platformě.
Cílem této práce je tak:
\begitems \style n
* Srovnat existující přístupy k facetovému vyhledávání, především pak z hlediska využití sémantických technologií.
* Navrhnout modul sémantického facetového vyhledávače, který bude umožňovat rozdělení vyhledávání a jeho vizualizace do samostatných modulů.
* Naimplementovat prototyp modulu sémantického vyhledávače.
\enditems