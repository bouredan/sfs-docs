\chap Implementace
V této kapitole si popíšeme postup implementace návrhu {\em sfs-api} z předešlé kapitoly \ref[navrh].
Zdůvodníme si některé rozhodnutí při vývoji a použité knihovny.

\sec Popis implementace
Při implementaci jsme

\sec Použité knihovny
Při vývoji jsme se snažili, aby knihovny měly co nejméně závislostí, aby byly nenáročné na místo a neinstalovali uživateli závislosti, které nepotřebuje nebo nechce.
Zde si popíšeme využité knihovny.
Všechny tři projekty jsou napsané v TypeScriptu, což je nadstavba JavaScriptu, která jej rozšiřuje o statické typování a předchází tak spoustě chyb.

\secc sfs-api
Na balíček {\em sfs-api} jsme si vystačili pouze s třemi závislostmi. 
Využíváme knihovnu {\em fetch-sparql-endpoint}\fnote{https://www.npmjs.com/package/fetch-sparql-endpoint}.
Ta nám zjednodušuje volání \glref{SPARQL} endpointu a serializaci příchozích dat do formátu RDFJS. 
Vybrali jsme ji hlavně proto, že je skutečně jednoduchá, neimplementuje nepotřebné věci navíc a je udržována poměrně aktuální.

Balíček datového modelu RDFJS je druhou závislostí.
{\em sfs-api} využívá ještě knihovnu {\em sparqljs}\fnote{https://www.npmjs.com/package/sparqljs} ke parsování SPARQL dotazů do JavaScript objektů.
Skrze tyto objekty můžeme pak query lépe upravovat.

\secc react-sfs
Knihovna {\em react-sfs} má jako závislost pouze {\em sfs-api}.

\secc sfs-react-demo
Jako součást zadání této práce bylo rozhodnuto, že na implementaci vyhledávače bude použit JavaScript framework React. 
K vytvoření projektu vyhledávače, tedy {\em sfs-react-demo} jsme použili doporučenou metodu vytváření React projektů {\em create-react-app}. 
Ta vytvoří předpřipravený projekt s minimální konfigurací a nejužitečnějšími závislostmi.

Abychom se nemuseli zaobírat UI prvky prototypu, využili jsme knihovnu MUI (také známou také jako Material-UI) a použili připravené komponenty z ní.
Druhá knihovna, kterou využíváme je fetch-sparql-endpoint \cite[fetch-sparql-endpoint]. 
Ta nám zjednodušuje volání \glref{SPARQL} endpointu a serializaci příchozích dat do formátu RDFJS. 
Vybrali jsme ji hlavně proto, že je skutečně jednoduchá, neimplementuje nepotřebné věci navíc a je udržována aktuální.

Za zmínku pak ještě stojí knihovna {\em react-virtuoso}, kterou využíváme na virtualizaci listů s výsledky a možnostmi facetů.
Virtualizace listu znamená, že se renderuje vždy jen část listu, která je zobrazená a ne ostatní položky, které se zrovna nezobrazují.
List totiž může obsahovat velmi velké množství dat a renderování všech položek by mohlo znatelně zpomalit naší aplikaci.

