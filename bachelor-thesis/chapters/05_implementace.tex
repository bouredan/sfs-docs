\chap [implementace] Implementace
V této kapitole si popíšeme implementaci návrhu z předešlé kapitoly \ref[navrh].
Zdůvodníme si některé rozhodnutí při vývoji a použité knihovny.

\sec Použité knihovny
Při vývoji jsme se snažili, aby knihovny měly co nejméně závislostí, aby byly nenáročné na místo a neinstalovaly uživateli závislosti, které nepotřebuje nebo nechce.
Zde si popíšeme využité knihovny.
Všechny tři projekty jsou napsané v jazyku TypeScriptu, což je nadstavba jazyku JavaScript, která jej rozšiřuje o statické typování a předchází tak spoustě chyb.

Na balíček {\em sfs-api} jsme si vystačili pouze s třemi závislostmi. 
Využíváme knihovnu {\em fetch-sparql-endpoint}\urlnote{https://www.npmjs.com/package/fetch-sparql-endpoint}.
Ta nám zjednodušuje volání \glref{SPARQL} endpointu a serializaci příchozích dat do formátu RDFJS. 
Vybrali jsme ji hlavně proto, že je skutečně jednoduchá, neimplementuje nepotřebné věci navíc a je udržována poměrně aktuální.

Balíček datového modelu RDFJS je druhou závislostí.
{\em sfs-api} využívá ještě knihovnu {\em sparqljs}\urlnote{https://www.npmjs.com/package/sparqljs} ke parsování SPARQL dotazů do JavaScript objektů.
Skrze tyto objekty můžeme pak query lépe upravovat.

Knihovna {\em react-sfs} má jako závislost pouze {\em sfs-api}.

Jako součást zadání této práce bylo rozhodnuto, že na implementaci vyhledávače bude použit JavaScript framework React. 
K vytvoření projektu vyhledávače, tedy {\em sfs-react-demo} jsme použili doporučenou metodu vytváření React projektů {\em create-react-app}. 
Ta vytvoří předpřipravený projekt s minimální konfigurací a nejužitečnějšími závislostmi.

Abychom si usnadnili vytváření \glref{UI} vyhledávače, využili jsme populární knihovnu MUI (známou též jako Material-UI) a použili připravené komponenty z ní.

Za zmínku pak ještě stojí knihovna {\em react-virtuoso}, kterou využíváme na virtualizaci listů s výsledky a možnostmi facetů.
Virtualizace listu znamená, že se renderuje vždy jen část listu, která je zobrazená a ne ostatní položky, které se zrovna nezobrazují.
List totiž může obsahovat velmi velké množství dat a renderování všech položek by mohlo znatelně zpomalit naší aplikaci.

\sec Implementační detaily
Implementace probíhala poměrně podle plánu s minimálními odchylkami od návrhu.
V této sekci si však zmíníme některé detaily, které jsou zajímavé a které byly třeba zpočátku vymyšleny jiným způsobem.

Původně bylo zamýšleno, že při konfiguraci budou {\ssr SfsApi} předány pouze objekty rozhraní {\ssr FacetConfig}, se 
specifikovaným typem facetu a {\ssr SfsApi} teprve při konstrukci vytvoří jednotlivé instance podtříd.
Dobře by se na toto dal využít {\em Factory pattern}.
To se však ukázalo jako těžko proveditelné, pokud chceme zachovat možnost implementovat si vlastní podtřídu {\ssr Facet}.
Třída {\ssr SfsApi} by těžko věděla jakou třídu zkonstruovat pro custom facety.
Lepší tedy pro nás bylo předat třídě {\ssr SfsApi} rovnou instance podtříd třídy {\ssr Facet}.

Dalším detailem, který se lehce proměnil bylo skládání constraints z {\ssr SfsApi} a ostatních facetů do query facetu.
Zpočátku se tyto constraints přidávaly do query až v třídě {\ssr SfsApi}, nakonec se však vytvořila metoda v {\ssr SfsApi} 
pro získání constraints a skládání se tak přesunulo do samotných facetů.
To umožňuje větší variabilitu pro implementaci vlastního facetu.
Uživatel si může sám rozhodnout jak chce naložit s aktuálními constraints ostatních facetů či celého facetového vyhledávání.

S constraints souvisí i další detail, který popíšeme.
S částmi \glref{SPARQL} query jsme chtěli co nejvíce pracovat jako s objektami dle typů ze {\em sparqljs}.
Výhodami toho je menší chybovost díky typovosti těchto objektů a zároveň jasnější a srozumitelnější kód.
Problém však nastal při implementování constraints facetu.
Pokud totiž měl facet v {\ssr FacetConfig} jako {\ssr predicate} \glref{IRI}, které bylo složeno z prefixu definovaným v {\ssr SfsApiConfig}, nastal 
problém jak tento constraints reprezentovat.
Nejvhodnějším řešením se nakonec zdálo být používat pouze jeden {\ssr Parser}{\fnote{Třída {\ssr Parser} je součástí knihovny {\em sparqjs} a vygeneruje JavaScript objekt z textové reprezentace \glref{SPARQL} query s použitím předaných prefixů.} k sestrojování objektů.
Ten je v {\ssr SfsApi} a má tedy přístup k prefixům ze {\ssr SfsApiConfig}.

