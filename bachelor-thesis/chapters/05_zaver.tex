\chap Závěr
V práci jsme se seznámily s pojmem sémantický web a popsali si jeho klíčové technologie. 
Poté jsme si představily problematiku facetového vyhledávání, kde jsme si zadefinovali jak by to měl vypadat faceotový vyhledávač a zanalyzovali způsoby jak jej implementovat.
S těmito poznatky jsme navrhli a následně implementovali prototyp sémantického facetového vyhledávače. Celý proces návrhu jsme zdokumentovali.
Postupně jsme tak splnili všechny cíle této práce.

Implementovaný prototyp je však stále doopravdy jen prototypem a do finální podoby má daleko. 
To se ale v budoucnosti změní, jelikož bude dále rozvíjen jako součást bakalářské práce a následně publikován jako open source balíček. 
Pokud bude finální vyhledávač implementovaný dobře, mohl by se začít používat jako primární dostupné řešení facetového vyhledávání pro sémantická data.
Nahradit by mohl i vyhledávač pro Sémantický slovník pojmů udržovaný Ministerstvem Vnitra České Republiky (MVČR) ve společnosti s \glref{FEL} \glref{ČVUT}, který je nevyhovující.

Ačkoliv vývoj sémantického webu v minulém desetiletí spíše stagnoval a nepodařilo se mu rozšířit tak jak si jeho autoři přáli, v posledních letech se to značně mění.
S průlomy v posledních v oblasti umělé inteligence či internetu věcí se opět o sémantickém webu mluví jako o jedné z přicházejících klíčových technologií ve vývoji webu a internetu a šlo by tak očekávat jeho výrazný vývoj přímo v následující letech.
Jestli se predikce potvrdí ukáže čas, ale je možné, že i tato práce pak bude využívána v nově vzniklých aplikacích jako řešení sémantického facetového vyhledávání.
