\chap Závěr
V práci jsme se seznámily s pojmem sémantický web a popsali si jeho klíčové technologie. 
Poté jsme si představily problematiku facetového vyhledávání, kde jsme si zadefinovali jak by to měl vypadat facetový vyhledávač a zanalyzovali způsoby jak jej implementovat.
S těmito poznatky jsme navrhli a následně implementovali knihovny pro implementaci facetového vyhledávače a s jejich pomocí i vyhledávač samotný.
Postupně jsme tak splnili všechny cíle této práce.

Pokud bude finální vyhledávač implementovaný dobře, mohl by se začít používat jako primární dostupné řešení facetového vyhledávání pro sémantická data.
Nahradit by mohl i vyhledávač pro Sémantický slovník pojmů udržovaný Ministerstvem Vnitra České Republiky (MVČR) ve společnosti s FEL ČVUT, který je nevyhovující.

Ačkoliv vývoj sémantického webu v minulém desetiletí spíše stagnoval a nepodařilo se mu rozšířit tak jak si jeho autoři přáli, v posledních letech se to však mění.
S průlomy v posledních v oblasti umělé inteligence či internetu věcí se opět o sémantickém webu mluví jako o jedné z přicházejících klíčových technologií ve vývoji webu a internetu a šlo by tak očekávat jeho výrazný vývoj přímo v následující letech.
Jestli se predikce potvrdí ukáže čas, ale je možné, že i tato práce pak bude využívána v nově vzniklých aplikacích jako řešení sémantického facetového vyhledávání.
