\chap Závěr
V práci jsme se seznámili s pojmem sémantický web a popsali si jeho klíčové technologie.
Detailněji jsme si ukázali především technologie k ukládání a dotazování dat.

Dále jsme si popsali facetové vyhledávání a zároveň zadefinovali jak by měl vypadat správný facet.
Rozdělili jsme si typy facetů dle toho jak vypadají z pohledu uživatele.
Následně jsme zrešeršovali dostupná řešení facetového vyhledávání a vybrali dvě, každé s jiným přístupem.
Zaměřili jsme se také na srovnání přístupů k facetovému vyhledávání z hlediska využití sémantických dat, čímž jsme splnili první z cílů této práce.

S nabytými poznatky jsme se pustili do návrhu sémantického facetového vyhledávače.
Návrh jsme pojali spíše jako návrh knihovny SFS za pomoci které pak lehce implementujeme samotný vyhledávač.
Knihovnu jsme rozdělili na balíčky {\em sfs-api}, který je jádrem celé knihovny a {\em react-sfs}, který poskytuje funkce k jednoduché implementaci na platformě React.
Oba balíčky jsme publikovali v repozitáři Node.js balíčků {\em npm}.
Samotný vyhledávač je pak projekt {\em sfs-react-demo}, který tyto knihovny využívá a implementuje vizuální vrstvu vyhledávače.
Tímto jsme splnili druhý a třetí cíl této práce.

Abychom ověřili funkčnost vyhledávače implementovali jsme několik testů.
Hlavně jsme však vytvořený vyhledávač srovnali s již existujícím Prohlížečem sémantického slovníku pojmů 
udržovaným Ministerstvem Vnitra České Republiky (MVČR) ve spojení s FEL ČVUT.
S ohledem na výsledky autor práce doporučil, aby byl starý prohlížeč nahrazen novým.
Ten by používal knihovnu SFS a byl postaven na projektu {\em sfs-react-demo}.
Tímto srovnáním jsme splnili poslední cíl této práce.

Ačkoliv vývoj sémantického webu v druhé polovině minulého desetiletí spíše stagnoval a nepodařilo se mu možná rozšířit tak, jak si jeho autoři přáli, v posledních letech se to mění.
S průlomy v posledních letech v oblasti umělé inteligence či internetu věcí se opět o sémantickém webu mluví jako o jedné z dalších klíčových technologií ve vývoji webu a internetu.
Tyto oblasti mají totiž pro sémantický web spoustu využití a je tudíž možné, že v následujících letech opět zažije výrazný vývoj.
Jestli se predikce potvrdí ukáže čas, ale je možné, že i tato práce pak bude využívána v nově vzniklých aplikacích jako řešení sémantického facetového vyhledávání a pomůže tak adopci sémantického webu.

\sec Plány do budoucna
Aby měla knihovna SFS větší předpoklady k tomu být používaná jako primární řešení facetového vyhledávání pro sémantická data, je třeba doimplementovat ještě několik funkcí.

V první řadě to je stránkování. 
Virtualizované listy sice fungují dobře, ale ve velkém množství výsledků se lépe orientuje a pohybuje pomocí stránek.
Navíc by to bylo výhodnější z hlediska rychlosti a vytížení \glref{SPARQL} endpointu, jelikož bychom mohli dotazovat jen část výsledků pro danou stránku.

S tím souvisí i přidání \glref{URL} parametrů. 
V těch by totiž mohly být, kromě čísla aktuální stránky, také zvolené hodnoty facetů a dotaz na vyhledávání.
To by nám umožnilo vytvářet odkazy na stav vyhledáváče a jeho výsledků.

Určitě je také potřeba přidat podporu více typů facetů, jako například range facet či date facet. 
Jelikož je celý projekt open source, mohly by být tyto funkce implementovány už za pomoci dalších členů open source komunity.
