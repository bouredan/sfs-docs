\chap Závěr
V práci jsme se seznámili s pojmem sémantický web a popsali si jeho klíčové technologie.
Detailněji jsme se hlavně seznámili s technologiemi k ukládání a dotazování dat.

Poté jsme si popsali facetové vyhledávání a zároveň zadefinovali jak by měl vypadat správný facet.
Rozdělili jsme si typy facetů dle toho jak vypadají z pohledu uživatele.
Následně jsme zrešeršovali dostupná řešení facetového vyhledávání a vybrali dvě každé s jiným přístupem.
Zaměřili jsme se také na jejich srovnání z hlediska využití sémantických dat, čímž jsme splnili první z cílů této práce.

S nabytými poznatky jsme se pustili do návrhu sémantického facetového vyhledávače.
Návrh jsme pojali spíše jako návrh knihovny SFS za pomocí které, pak lehce implementujeme samotný vyhledávač.
Knihovnu jsme rozdělili na balíčky {\em sfs-api}, který je jádrem celé knihovny a {\em react-sfs}, který poskytuje funkce k jednoduché implementaci na platformě React.
Oba balíčky jsme publikovali v repozitáři Node.js balíčků {\em npm}.
Samotný vyhledávač je pak projekt {\em sfs-react-demo}, který tyto knihovny využívá a implementuje vizuální vrstvu vyhledávače.
Tímto jsme splnili druhý a třetí cíl této práce.

Abychom ověřili funkčnost vyhledávače implementovali jsme několik testů.
Hlavně jsme však vytvořený vyhledávač srovnali s již existujícím Prohlížečem sémantického slovníku pojmů 
udržovaným Ministerstvem Vnitra České Republiky (MVČR) ve spojení s FEL ČVUT.
S ohledem na výsledky srovnání jsme doporučili, aby byl starý vyhledávač nahrazen po domluvě a drobných úpravách nově vzniklým.
Zároveň jsme tímto srovnáním splnili poslední cíl této práce.

Jelikož se dle výsledků vyhodnocení zdá, že je knihovna SFS navržena dobře, mohla by se začít používat jako primární řešení facetového vyhledávání pro sémantická data.
K tomu je však nejspíše potřeba ještě doimplementovat další funkce, které jsme popsali v sekci \ref[plany-do-budoucna].
Jelikož je celý projekt open source, mohlo by se tak stát už za pomoci dalších členů open source komunity.

Ačkoliv vývoj sémantického webu v druhé polovině minulého desetiletí spíše stagnoval a nepodařilo se mu možná rozšířit tak jak si jeho autoři přáli, v posledních letech se to však mění.
S průlomy v posledních letech v oblasti umělé inteligence či internetu věcí se opět o sémantickém webu mluví jako o jedné z přicházejících klíčových technologií ve vývoji webu 
a internetu a šlo by tak očekávat jeho výrazný vývoj přímo v následující letech.
Jestli se predikce potvrdí ukáže čas, ale je možné, že i tato práce pak bude využívána v nově vzniklých aplikacích jako řešení sémantického facetového vyhledávání a pomůže tak rozšíření sémantického webu.
