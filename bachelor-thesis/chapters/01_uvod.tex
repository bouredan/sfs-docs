\chap Úvod
Málokterý vynález ovlivnil svět v takové míře jako vznik World Wide Web (zkráceně \glref{WWW} či web). 
Za poměrně krátkou dobu své existence se web rozšířil téměř do každé části našeho života a dnes si bez něj lze svět jen těžko představit. 
Oproti ostatním \glref{ICT} technologiím, které se často výrazně inovují a mění každých několik let, web funguje už 20 let téměř stejně.

To se však začíná měnit s příchodem sémantického webu, který zásadně ovlivňuje, jak přistupujeme k datům v internetu – místo relací mezi dokumenty přes hypertextové odkazy můžeme vytvářet relace mezi fakty. 
Svět lze tak mnohem lépe popsat a stává se pro nás srozumitelnější. 
Navíc jsou tyto relace jednoduše strojově čitelné, tudíž se stává srozumitelnější nejen pro nás, ale i pro stroje. 
Ty poté mohou nad těmito daty mnohem přesněji vyhledávat informace či vykonávat automatizace. 

Interakce se sémantickými data vyžaduje nové přístupy k ukládání, zpracování a vyhledávání dat. 
Právě vyhledáváním v sémantických datech se zabývá tato práce, konkrétně facetovým vyhledáváním. 
Facetové vyhledávání, tedy zatřídění vyhledaných výsledků do různých kategorií, je v současné době velmi rozšířené. 
Pomáhá nám upřesnit výsledky vyhledávání a najdeme jej například téměř v každém větším e-shopu. 

Přístupů k facetovému vyhledávání je více, ne všechny jsou však vhodné pro sémantická data. 
Nad sémantickými daty tak existuje velmi málo řešení facetového vyhledávání a ty existující mají své nedokonalosti.
Často jsou závislé na nějakou platformu nebo jsou velmi omezující.

Snahou této práce je navrhnout a implementovat sémantický facetový vyhledávač, který by byl jednoduše lepší.
Měl by tedy být nezávislý na platformě a co nejméně omezující, aby mohl být integrován do různých projektů.
K implementování facetového vyhledávače byla sice zvolena platforma React, ale jeho logika vyhledávání by měla být integrovatelná do jakékoliv JavaScript platformy.
Vyhledávač tak musí být rozdělen do modulů na samotné vyhledávání a jeho vizualizace.

Jedním z existujících řešení sémantického facetové vyhledávání je SPARQL Faceter, který je dostupný pouze na platformě AngularJS.
Toto řešení využívá stránka Prohlížeče sémantického slovníku pojmů Ministerstva vnitra České republiky (MVČR) na adrese \url{https://slovník.gov.cz/prohlížeč}.
Sémantický facetový vyhledávač implementovaný v této práci bude s tímto prohlížečem porovnán.
Vyhledávač tak musí umožňovat rozdělení vyhledávání a jeho vizualizace do samostatných modulů.
\nl\nl

\midinsert
Cílem této práce jsou:
\begitems
* Srovnat existující přístupy k facetovému vyhledávání, především pak z hlediska využití sémantických technologií.
* Navrhnout modul sémantického facetového vyhledávače, který bude umožňovat rozdělení vyhledávání a jeho vizualizace do samostatných modulů.
* Naimplementovat navržené řešení včetně vizualizačního modulu.
* Ověřit funkčnost řešení srovnáním s existujícím facetovým vyhledávačem na stránkách \url{https://slovník.gov.cz/prohlížeč}.
\enditems
\endinsert
