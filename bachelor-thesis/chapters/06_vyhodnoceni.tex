\chap [vyhodnoceni] Vyhodnocení
Tato kapitola se bude zabývat vyhodnocením implementovaného vyhledávače z předchozích kapitol \ref[navrh] a \ref[implementace].
Nejprve představíme testy, které jsme implementovali k ověření jeho funkčnosti.
Pak srovnáme vyhledávač SFS s vyhledávačem, který jsme v sekci \ref[sparql-faceter] nazvali jako prohlížeč MVČR.
Na konci srovnání pak bude na jeho základě doporučeno či nedopručeno nahrazení prohlížeče MVČR vyhledáváčem SFS.

\sec [testy] Testy
K otestování knihovny SFS jsme se rozhodli využít jednotkové testy.
Tyto testy však prozatím testují pouze balíček {\em sfs-api}, jelikož je hlavní částí celého vyhledávání.
Pro účely této práce se to zdá dostačující, ale do budoucna by bylo dobré zvýšit počet jednotkových testů a vytvořit testy i v projektech {\em react-sfs} a {\em sfs-react-demo}.

Testy {\ssr SfsApi} se prvně zaměřují na kontrolu konfigurace, jak můžeme vidět na příkladu níže.
\begtt
it('should throw an error on baseQuery without _id variable', () => {
    expect(() => {
      new SfsApi({
        endpointUrl: "https://dbpedia.org/sparql",
        facets: [],
        baseQuery: "SELECT DISTINCT ?x ?_label WHERE {?x ?y ?_label}",
        language: "en"
    });
  }).toThrowError("SfsApi baseQuery has to SELECT ?_id variable. " + 
   "Check documentation for more info.")
});
\endtt

U facetů pak testujeme například správné emitování eventů.
\begtt
it('emits FETCH_FACET_OPTIONS_SUCCESS event', done => {
  jest.spyOn(sfsApi, "fetchBindings")
	  .mockImplementation(() => {
	    const stream = new PassThrough();
	    stream.end();
	    return Promise.resolve(stream);
	  });
	sfsApi.eventStream.on("FETCH_FACET_OPTIONS_SUCCESS", event => {
	  done();
	});
	facet.refreshOptions();
});
\endtt

\sec [srovnani-s-existujicim-vyhledavacem] Srovnání s existujícím vyhledávačem
V rámci zadání práce máme porovnat implementovaný vyhledávač s existujícím prohlížečem MVČR.
Hlavní rozdíly těchto vyhledávačů jsou vypsány v tabulce \ref[srovnani-vyhledavacu-tabulka] a dále popsány v této sekci.

\medskip \label[srovnani-vyhledavacu-tabulka]
\ctable{lcc}{
\hfil  & Prohlížeč MVČR & Vyhledávač SFS \crl
Rychlost načtení stránky & 70~sekund & 3,9~sekund \cr
Vlastní vizuální podoba & NE & ANO \cr
Nezávislost na platformě & NE & ANO \cr
Možnost implementace vlastního facetu & NE & ANO \cr
Virtualizace listů & NE & ANO \cr
Volba jazyku & ANO, ale jen výsledky & NE \cr
}
\caption/t Srovnání vyhledávačů.
\medskip

Největší rozdíl mezi vyhledávači můžeme sledovat už při prvotním načtení stránky.
Zatímco prohlížeč MVČR z 10 měření načetl výsledky průměrně za {\sbf 70~sekund}, vyhledávač SFS je načetl průměrně za {\sbf 3,9~sekund}.
K tomuto je nutné říct, že je to především způsobeno množstvím přenesených dat.
Prohlížeč MVČR totiž ukazuje více informací o výsledcích vyhledávání a tím pádem se do něj přenáší téměř 140 MB dat, zatímco do vyhledávače SFS zhruba 10.5 MB.
Není tedy oprávněné toto brát jako výhodu naší implementace vyhledávače.
Je ovšem k zamyšlení, zda je žádoucí takové množství informací při načtení stránky načítat, jelikož čekat déle než minutu na načtení výsledků je z pohledu uživatele velmi nevhodné.

Rozdíly, ve kterých je však vyhledávač SFS rozhodně lepší jsou možnost vlastní vizuální podoby, nezávislost na platformě a možnost implementace vlastního facetu.
Vizuální podoba prohlížeče MVČR je totiž omezena použitou knihovnou SPARQL Faceter, která je však už 6 let stará a na jejím vzhledu je to znát.
Navíc je knihovna SPARQL Faceter navržena pouze pro platformu AngularJS a tudíž omezuje uživatele na použití této platformy.
Jak vypadá prohlížeč MVČR si můžeme prohlédnout na obrázku \ref[prohlizec-mvcr-page] nebo na jeho adrese \url{https://slovník.gov.cz/prohlížeč}.

Vyhledávač SFS nechává vizuální podobu čistě na uživateli a dá se využít na kterékoliv JavaScript platformě, tudíž své uživatele nijak neomezuje.
K tomu přispívá i možnost implementace vlastního facetu pro případy nepokryté knihovnou SFS.
Vzhled vyhledávače SFS je na obrázku \ref[vyhledavac-sfs-page] nebo si lze lokálně spustit projekt {\em sfs-react-demo}\fnote{Pokyny ke spuštění můžeme najít v příloze \ref[elektronicka-priloha].}.

\midinsert \label[prohlizec-mvcr-page]
\picw=14cm \cinspic images/prohlizec-mvcr-page.png
\caption/f Stránka prohlížeče MVČR.
\endinsert

\midinsert \label[vyhledavac-sfs-page]
\picw=14cm \cinspic images/vyhledavac-sfs-page.png
\caption/f Stránka vyhledávače SFS.
\endinsert

Vyhledávač SFS je také lepší v tom, že využívá virtualizované listy pro výsledky i možnosti facetů a vyžaduje tak méně výkonu.
Naopak však zatím nepodporuje volbu jazyka jako prohlížeč MVČR.
U něho se tato možnost však aplikuje pouze na výsledky a navíc se nijak neukládá, tudíž je nutné při každém obnovení stránky znovu tuto možnost zvolit.

Pak je zde několik problematických částí prohlížeče MVČR, které v tabulce nejsou zmíněny.
Jsou to například nicneříkající názvy glosářů v příslušném facetu, nesmyslné řazení výsledků a možností facetů nebo špatné stylování tabulky s výsledky, ze které není vidět celý sloupec s glosářy.
I tyto části lze vidět na obrázku \ref[prohlizec-mvcr-page].
Za zmínku stojí také otravné načítací kolečko výsledků, které vždy posune celou tabulku s výsledky pod sebe.

Na základě tohoto srovnání doporučuje autor této práce nahradit prohlížeč MVČR novým vyhledávačem za použití knihovny SFS. 
Nový vyhledávač by byl založený na vyhledávači použitým v tomto srovnání (tedy projekt {\em sfs-react-demo}), ale musel by nejspíše projít drobnými úpravami, aby plně splňoval představy správců stávajícího prohlížeče.
