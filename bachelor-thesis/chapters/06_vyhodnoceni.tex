\chap Vyhodnocení
Tato kapitola se zabývá vyhodnocením implementovaného vyhledávače z předchozích kapitol \ref[navrh] a \ref[implementace].
Nejdříve představíme testy, které jsme implementovali k ověření jeho funkčnosti.
Hlavní sekcí však bude srovnání s existujícím vyhledávačem, který jsme v sekci \ref[sparql-faceter] nazvali jako prohlížeč MVČR\urlnote{https://slovník.gov.cz/prohlížeč}.
Poté představíme plány do budoucna.

\sec [testy] Testy
TODO až budou hotové testy v sfs-api

\sec [srovnani-s-existujicim-vyhledavacem] Srovnání s existujícím vyhledávačem
V rámci zadání práce máme porovnat implementovaný vyhledávač s existujícím prohlížečem MVČR.
To bude náplní této sekce.
Dle výsledků srovnání se také zamyslíme nad doporučením nahrazení stávajícího prohlížeče naším novým vyhledávačem.

TODO zmínit rychlost načítání, virtualizace, oddělené načítání, jazyk, nezávislost na vizuální podobě, možnost implementace vlastního facetu, lepší labely

\sec [plany-do-budoucna] Plány do budoucna
Ačkoliv je vyhledávač hotový a funguje, dle předcházejících sekcí \ref[testy] a \ref[srovnani-s-existujicim-vyhledavacem], dobře, stále je několik funkcí, které by bylo dobré v budoucnu implementovat.

V první řadě to je stránkování. 
Virtualizované listy sice fungují dobře, ale stránkování by mohlo lepší na navigaci ve velkém množství výsledků.
Dále by bylo příhodné přidat do \glref{URL} parametrů zvolené hodnoty facetů a dotaz na vyhledávání .
To by nám umožnilo vytvářet odkazy na stav vyhledáváče a tedy i jeho výsledků.

Další věci ke zlepšení do budoucna jsou spíše vývojářské.
Například třída {\ssr CheckboxFacet} z {\em sfs-api} by mohla být generická místo definovaného typu hodnot {\ssr string[]} nebo by se dalo vylepšit rozhraní {\em sfs-api} skrz třídu {\ssr EventStream}.
Při používání třídy {\ssr EventStream} totiž může dojít k tomu, že se při více asynchronních požadavcích najednou změní pořadí jejich odpovědí a komponenty by mohly ukazovat špatný stav.

