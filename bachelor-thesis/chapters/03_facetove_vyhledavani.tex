\chap Facetové vyhledávání
V této kapitole si popíšeme, co je facetové vyhledávání a k čemu se primárně využívá. Zanalyzujeme a srovnáme pak různé přístupy k implementaci
facetového vyhledávání, především z hlediska využití sémantických technologií. Abychom získali přehled o používaných řešení facetového vyhledávání, zanalyzujeme 
poskytované Facet Search \glref{API}s největších společností v této oblasti jako Elastic či Solr.

\sec Popis
\medskip
\picw=15cm \cinspic images/facets-explained.jpeg
\caption/f Ukázka facetů s vysvětlivkami \cite[facets-explained-image].
\medskip
Facetové vyhledávání je zatřídění vyhledaných výsledků do různých kategorií (facetů) dle kterých se dá sada výsledků dále filtrovat.
Dá se ním tak obohatit každé vyhledávání, ale často bývá spojeno s fulltextovým vyhledáváním\rfc{možná někde vysvětlit co to je}, aby uživateli umožnilo jeho dotaz dále upřesnit.
Hojně se využívá třeba v e-commerce sektoru, kde podle studie Nielsen Norman Group (NNG) z roku 2018 jsou e-shopy bez facetového vyhledávání výjimkou \cite[nng-study].
Jelikož není definovaný žádný standard facetového vyhledávání, zadefinujeme si, co by měl takový modul facetového vyhledávání splňovat:
\begitems
* facet obsahuje hodnotu pro každý výsledek ze sady výsledků
* jednotlivé facety lze kombinovat mezi sebou
* mezi kritérii facetů platí logický AND (ne pouze OR), tzn. aby se výsledek objevil v sadě výsledku, musí vyhovět všem aktivním facetům
* hodnoty facetů ukazují počet výsledků, které aplikování facetu s danou hodnotou v aktuálním stavu vrátí
* hodnoty facetů, které by vrátily prázdnou sadu výsledků se nezobrazují nebo jsou \uv{disabled}\fnote{Jejich \glref{HTML} ovládací prvek má atribut disabled.}
\enditems

\sec Typy facetů
Facety si můžeme rozřadit dle toho jakým způsobem se volí jejich hodnoty. V této práci budeme tyto kategorie nazývat jako typy facetů.

  \secc Select facet
  Facet s možností volby nejvýše jedné hodnoty podle které je pak sada výsledků filtrována. Ovládacím prvkem bývá select element.

  \secc Checkbox facet
  Facet s možností volby více hodnot skrz zaškrtávání checkboxů.

  \secc Range facet
  Facet pro číselná data s možností nastavení rozsahu. Ovládacím prvkem bývá posuvník (input element s hodnotou atributu type range).

  \secc Bucket facet
  Podobné jako range facet, akorát se neovládá posuvníkem, ale jsou nadefinovány rozsahy, do kterých se pak výsledky roztřídí.
		
\sec Srovnání přístupů
Řešení jak implementovat facetové vyhledávání je více. 
Obecně je lze rozdělit podle toho, kde dochází k filtraci výsledků aktivními facety, tedy jestli na straně klienta či na straně serveru.
U implementace facetového vyhledávání je také nutné myslet na to jakým způsobem se budou plnit data facetů. Nejčastěji se setkáváme s tím, že se
hodnoty facetů posílají ve stejné response jako sada výsledků. 

  \secc Filtrování na straně klienta
  Při filtrování na klientu nám server vždy zašle celou výsledkovou sadu, kterou při zpracování vyfiltrujeme dle aktivních facetů a zobrazíme jen vyfiltrované výsledky.
  To znamená, že sadu výsledků stačí teoreticky zaslat jen jednou a případně lehce \uv{zacachovat}.
  Nemusíme tak také řešit zasílaní facetů v komunikaci se serverem a ta se tak stává vcelku přímočará a přehledná.
  S tím je však ale spojená i hlavní nevýhoda – sada výsledků může být velká, což znamená velké množství dat, které musíme přenést internetem, uložit na klientovi a následně vyfiltrovat.
  To může při pomalejším spojení nebo malém výpočetním výkonu na klientovi trvat delší dobu. 

  \secc Filtrování na straně serveru
  Filtrování na serveru většinu \uv{tvrdé dřiny} přenáší na server, což přirozeně zvyšuje jeho zatížení.
  Musíme serveru spolu s požadavkem předat jaká data chceme (aktivní facety) a on nám musí zpátky poslat již vyfiltrovanou sadu výsledků a nové hodnoty pro tyto facety (společně s počtem jejich výskytů).
  Zároveň to ale znamená, že pokud je server správně implementován, může přidat filtrování už do dotazu do databáze a celý proces tak výrazně zrychlit.
  Pokud k tomu ještě přidáme fakt, že server vrací vyloženě jen data, které požadujeme a kterých velikost tak může být značně nižší (a přenos rychlejší) jedná se tak jednoznačně o rychlejší metodu.
  Zároveň je toto řešení i spolehlivější a lépe škálovatelné.
	
\sec Převedení do sémantického světa
Jak jsme zmínili v minulé kapitole, sémantické data jsou dotazovány pomocí \glref{SPARQL}. 
V praxi to znamená, že k získání dat musíme zkonstruovat dotaz v syntaxi \glref{SPARQL} a poslat požadavek na publikovaný \glref{SPARQL} endpoint.
Jelikož se tento endpoint dá provolit i z klienta, nepotřebujeme vůbec server pro facetový vyhledávač.
Vyplývá nám z toho, že pro sémantická data je lepší použít filtrování už jako součást SPARQL dotazu.
Při konstrukci dotazu je nutné myslet, ale i na to, aby se nám vrátila data s hodnotami facetů. 
Toho lze docílit použití klauzule OPTIONAL.
Výsledné řešení pak bude něco mezi oběma popsanými přístupu v minulé sekci.
