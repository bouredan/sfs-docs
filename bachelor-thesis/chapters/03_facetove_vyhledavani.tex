\chap Facetové vyhledávání
V této kapitole si nejprve popíšeme, co je facetové vyhledávání a jak může vypadat z pohledu uživatele. 
Poté si zanalyzujeme různé přístupy k implementaci facetového vyhledávání díky rešerši populárních řešení.
Nakonec tyto přístupy srovnáme z hlediska využití sémantických technologií.


\sec [popis-facetoveho-vyhledavani] Popis

\rfc{je ok začít sekci obrázkem?}
\medskip
\picw=15cm \cinspic images/facets-explained.jpeg
\caption/f Ukázka facetů s vysvětlivkami \cite[facets-explained-image]
\medskip

Facetové vyhledávání je zatřídění vyhledaných výsledků do různých kategorií (facetů) dle kterých se dá sada výsledků dále filtrovat.
Dá se jím tak obohatit každé vyhledávání, ale často bývá spojeno s fulltextovým vyhledáváním\fnote{Fulltextové vyhledávání je technika hledání textu dle zadání uživatele v elektronických dokumentech a databázích\cite[fulltextove-vyhledavani].\rfc{je ok mit zdroj na fulltextové vyhledávání z wikipedie?}}, aby uživateli umožnilo jeho dotaz dále upřesnit.
Hojně se využívá třeba v e-commerce sektoru, kde podle studie Nielsen Norman Group (NNG) z roku 2018 jsou e-shopy bez facetového vyhledávání výjimkou \cite[nng-study].

Facety bývají částo znázorněny webovými elementy select či checkbox nebo také posuvníky.
Omezením, podle kterých facet filtruje výsledky a možnosti ostatních facetů po výběru hodnoty budeme v této práci nazývat \uv{constraints}.

Jelikož není definovaný žádný standard facetového vyhledávání, zadefinujeme si, co by měl takový modul facetového vyhledávání splňovat:
\begitems
* Facet obsahuje možnosti pro každý unikátní výsledek ze sady výsledků.
* Jednotlivé facety lze kombinovat mezi sebou.
* Po výběru hodnoty facetu se musí aktualizovat možnosti ostatních facetů.
* Mezi omezeními výsledků dle hodnot facetů platí logický {\ssr AND}, tzn. aby se výsledek objevil v sadě výsledku, musí vyhovět všem aktivním facetům.
* Možnosti facetů ukazují počet výsledků, které aplikování facetu s danou hodnotou v aktuálním stavu vrátí.
* Možnosti facetů, které by vrátily prázdnou sadu výsledků se nezobrazují nebo jsou \uv{disabled}\fnote{Jejich \glref{HTML} ovládací prvek má atribut disabled.
\enditems



\sec[typy-facetu] Typy facetů
Facety si můžeme rozřadit dle toho jakým způsobem se volí jejich hodnoty z pohledu uživatele.
V této práci budeme tyto kategorie nazývat jako typy facetů.

\secc Select facet
Facet s možností volby nejvýše jedné hodnoty, podle které je pak sada výsledků filtrována. 
Ovládacím prvkem bývá select element.

\secc Checkbox facet
Facet s možností volby více hodnot skrz zaškrtávání checkboxů.

\secc Range facet
Facet pro číselná data s možností nastavení rozsahu. 
Ovládacím prvkem bývá posuvník (input element s hodnotou atributu type range).

\secc Bucket facet
Podobné jako range facet, ale neovládá se posuvníkem, ale jsou nadefinovány a pojmenovány rozsahy, dle kterých se pak dají výsledky filtrovat.

\secc Text facet
Podobný fulltextovému vyhledávání.
Filtruje výsledky dle zadaného textu.


\sec[analyza-pristupu-k-facetovemu-vyhledavani] Analýza přístupu k facetovému vyhledávání
Abychom získali lepší přehled o facetovém vyhledávání, zanalyzujeme si dvě jeho řešení, každé s odlišným přístupem k facetovému vyhledávání.
Tyto řešení byly do práce vybrány jako ukázkové příklady po rešerši dostupných knihoven pro facetové vyhledávání.
Nejdříve zanalyzujeme facetové vyhledávání od Elasticsearch, které je typickým řešení facetového vyhledávání pro běžná data.
Poté si rozebereme sémantický facetový vyhledávač SPARQL Faceter.

\secc Elasticsearch
Elasticsearch je open source\fnote{Kvůli nedávné změně licence už technicky není úplně open source, stále má však veřejný kód a od založení fungoval jako open source.} nástroj pro fulltextové vyhledávání nad různými daty\cite[elasticsearch].
Má mnoho funkcí a možností, nás však zajímá pouze je část pro práci s facety.
\begtt
{
  "query": "park",
  "facets": {
    "states": [
      {
        "type": "value",
        "name": "top-five-states",
        "sort": { "count": "desc" },
        "size": 5
      }
    ]
  }
}
\endtt 
Výše můžeme vidět ukázkový požadavek na vyhledávání query \uv{park} společně s definováním facetu pro další filtrování\cite[elasticsearch-facet-api].
Facet je definován jménem \uv{top-five-states} a požaduje 5 možností řazených sestupně dle počtu výskytů.
Očekáváme tedy, že možnosti facetů budou poslány společně s výsledky vyhledávání.
\begtt
{
  ## odpověď obsahuje více polí, která však pro nás nejsou důležitá
  "results": [
    ## výsledky které vyhovují query "park"
  ]
  "facets": {
    "states": [
      {
        "type": "value",
        "name": "top-five-states",
        "data": [
          {
            "value": "California",
            "count": 8
          },
          {
            "value": "Alaska",
            "count": 5
          },
          {
            "value": "Utah",
            "count": 4
          },
          {
            "value": "Colorado",
            "count": 3
          },
          {
            "value": "Washington",
            "count": 3
          }
        ]
      }
    ]
  }
}
\endtt
Zde pak máme odpověď na požadavek.
Možnosti facetů jsou strukturovány vždy jako hodnota a počet výskytů.

\secc [sparql-faceter] SPARQL Faceter
Dalším příkladem přístupu k facetovému vyhledávání je modul na sémantické facetové vyhledávání od výzkumné skupiny Semantic Computing (SeCo)\cite[seco] s názvem SPARQL Faceter\cite[sparql-faceter-docs].
Balíček tohoto modulu se jmenuje {\em angular-semantic-faceted-search} a je to knihovna pro JavaScript framework Angular.
Tato knihovna je pro nás důležitá ze dvou důvodů.
Zaprvé je to jedno z mála řešení facetového vyhledávání pro sémantická data, tudíž může být inspirací pro náš návrh.
Zadruhé je touto knihovnou implementován sémantický facetový vyhledávač, se kterým se bude porovnávat i vyhledávač vzniklý v rámci této práce.
Porovnání těchto vyhledávačů se budeme věnovat v sekci \ref[srovnani-s-existujicim-vyhledavacem], tudíž nebudeme v této sekci \uv{zabrušovat}\fnote{Abychom z toho nemuseli pak zase \uv{vybrušovat}.} do úplných implementačních detailů, ale pouze se podíváme, jakým způsobem se posílají požadavky na výsledky a možnosti facetů.
Zmíněný existující sémantický facetový vyhledávač\urlnote{https://slovník.gov.cz/prohlížeč} budeme dále v práci nazývat jako prohlížeč MVČR.

Jako příklad na kterém budeme SPARQL Faceter analyzovat jsme si příhodně vybrali prohlížeč MVČR, který má dva facety.
Požadavky, které posílá SPARQL Faceter jsou prosté \glref{HTTP} POST či GET požadavky na \glref{SPARQL} endpoint\fnote{V tomto případě \url{https://slovník.gov.cz/sparql}.}.
Samotnou \glref{SPARQL} query pak skládá právě knihovna SPARQL Faceter na klientské straně.

\medskip \label[prohlizec-mvcr-network-requests]
\picw=15cm \cinspic images/prohlizec-mvcr-network-requests.png
\caption/f Asynchronní požadavky prohlížeče MVČR při načtení stránky
\medskip

Obrázek \ref[prohlizec-mvcr-network-requests] ukazuje asynchronní požadavky, které se odešlou při načítání stránky prohlížeče MVČR.
Z analýzy asynchronních požadavků můžeme vidět, že SPARQL Faceter posílá požadavek na možnosti facetu pro každý facet zvlášť.
Požadavek na výsledky se pošle také zvlášť.
To je velký rozdíl oproti Elasticsearch, kde se dotaz na možnosti facetů posílal společně s dotazem na výsledky.
Zároveň se posílá ještě jeden požadavek na zjištění počtu všech výsledků, který se zobrazuje u facetů pro žádnou nevybranou možnost.
Všechny tyto čtyři požadavky se posílají při každé změně facetu. 
Jejich \glref{SPARQL} query se však samozřejmě liší podle aktuálního stavu.

		
\sec Srovnání přístupů z hlediska sémantických technologií
TODO moc nevím jak tuto sekci pojmout\rfc{jak tuto sekci pojmout?}