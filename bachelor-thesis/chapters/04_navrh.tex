\chap[navrh] Návrh
\rfc{dodelat}
Výsledkem této práce je sémantický facetový vyhledávač na platformě React.
Tato kapitola se bude zabývat jeho návrhem.
Nejprve navrhneme jak rozdělit moduly vyhledávače, poté popíšeme architekturu jednotlivých modulů a komunikaci mezi jednotlivými facety.


\sec Moduly
Při návrhu je nutné dbát na to, aby byl vyhledávač rozdělený na moduly vyhledávání a jeho vizualizace.
Toto rozdělení nám umožní, aby byl modul vyhledávání zcela nezávislý na použité platformě.
Realizovat takový modul jde pak dvěmy způsoby - buďto jako samostatná serverová služba nebo jako modul bez využití konstruktů použité platformy, v našem případě React.
Jelikož lze získávat sémantická data skrze \glref{SPARQL} endpoint, který lze lehko volat přímo z prohlížeče, dává v našem případě větší smysl jít cestou modulu, který bude využíván přímo na frontendu.

Tento modul bude pojmenován {\ssr sfs-api} a bude koncipován tak, aby to bylo samostatná knihovna, kterou lze využít v jakémkoliv Javascript frameworku či prostředí.
Jelikož však chceme implementovat vyhledávač primárně na platformě React, vytvoříme ještě jednu knihovnu jménem {\ssr react-sfs}, která bude usnadňovat použití {\ssr sfs-api} v React aplikacích.
Výsledkem tak budou 3 projekty, knihovny {\ssr sfs-api} a {\ssr react-sfs} a finální projekt {\ssr sfs-react-demo}, který bude finálním sémantickým facetovým vyhledávačem a zároveň tak ukázkovým příkladem použití těchto knihoven.
Knihovny {\ssr sfs-api} a {\ssr react-sfs} budou publikovány v repozitáři NPM\fnote{https://www.npmjs.com}, který je primárním repozitářem Javascriptových balíčků.

Součástí publikovaných knihoven nejsou žádné komponenty či web elementy, které by přímo definovaly vizuální podobu samotných facetů či vyhledávání. 
Záměrně tuto část necháváme na uživateli, abychom ho nijak neomezovali a nevnucovali mu vizuální podobu, která se mu nemusí líbit.
Obě knihovny však poskytují rozhraní pro tyto web elementy, lze tedy využít i samostatně {\ssr sfs-api} pro jinou platformu než React.


\sec Architektura


\sec Popis eventů


\rfc{pridat listing label}
\begtt 
export const sfsApi = new SfsApi({
  endpointUrl: "https://xn--slovnk-7va.gov.cz/sparql",
  facets: [glosaryFacet, subClassOfFacet],
  queryTemplate:
    `SELECT DISTINCT ?_id ?_label 
WHERE 
  { ?_id a <http://www.w3.org/2004/02/skos/core#Concept> .
    FILTER isIRI(?_id)
    OPTIONAL
      { ?_id rdfs:label ?rdfsLabel 
        FILTER langMatches(lang(?rdfsLabel), "${language}")
      }
    OPTIONAL
      { ?_id skos:prefLabel ?prefLabel 
        FILTER langMatches(lang(?prefLabel), "${language}")
      }
      BIND(coalesce(?rdfsLabel, ?prefLabel, ?_id) AS ?_label) 
  }
ORDER BY ASC(?_label)`,
  prefixes: {
    rdfs: "http://www.w3.org/2000/01/rdf-schema#",
    skos: "http://www.w3.org/2004/02/skos/core#",
    dct: "http://purl.org/dc/terms/",
    dbp: "http://dbpedia.org/property/",
    dbo: "http://dbpedia.org/ontology/",
  },
  language,
});
\endtt

Facet je reprezentován rozhraním {\ssr Facet} a má své unikátní facetId, dle kterého je dále identifikován.
Jednotlivé typy facetů pak toto rozhraní rozšiřují a implementují jeho metodu \uv{generateSparql}, která se volá při sestavení výsledného \glref{SPARQL} dotazu.
Vracet by měla část \glref{SPARQL} týkající se tohoto facetu.
Podpora dalších typů je plánovaná s dalším vývojem vyhledávače. 
Tento design umožňuje uživateli vytvořit si vlastní typ facetu (implementováním interfacu Facet) pro případy nepokryté knihovnou, s tím, že ho ale stále může kombinovat s ostatními facety.

K připojení jednotlivých facetů používáme modul {\ssr react-sfs}. 
Ten za pomoci implementovaného React hooku spojí prezentační vrstvu se stavem facetu a je tak bodem komunikace s druhým modulem.
K tomu se používá hlavně subscriber pattern, kde je identifikátorem právě ono facetId.

\sec Technologie a knihovny
Jako součást zadání této práce bylo rozhodnuto, že na implementaci prototypu bude využit javascriptový framework React. 
Ten je v současné době velmi populární a pro naše potřeby vhodný, díky jeho modularizaci. 
Použili jsme tedy doporučovanou metodu create-react-app k vytvoření základních modulů. 
Abychom se nemuseli zaobírat UI prvky prototypu, využili jsme knihovnu MUI (dříve známá také jako Material-UI) a použili připravené komponenty z ní.
Druhá knihovna, kterou využíváme je fetch-sparql-endpoint \cite[fetch-sparql-endpoint]. 
Ta nám zjednodušuje volání \glref{SPARQL} endpointu a serializaci příchozích dat do formátu RDFJS. 
Vybrali jsme ji hlavně proto, že je skutečně jednoduchá, neimplementuje nepotřebné věci navíc a je udržována aktuální.
Všechen kód je napsaný v Typescriptu, což je nadstavba Javascriptu, která jej rozšiřuje o statické typování a předchází tak spoustě chyb.
