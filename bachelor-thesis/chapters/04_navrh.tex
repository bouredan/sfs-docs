\chap[navrh] Návrh
\rfc{míchám v celé kapitole budoucí čas s přítomným, je to ok?}
Výsledkem této práce je sémantický facetový vyhledávač na platformě React.
Tato kapitola se bude zabývat jeho návrhem.
Nejprve navrhneme architekturu vyhledávače, poté popíšeme jednotlivé části návrhu.


\sec Architektura
Při návrhu architektury je nutné dbát na to, aby byl vyhledávač rozdělený na moduly vyhledávání a jeho vizualizace.
Toto rozdělení nám umožní, aby byl modul vyhledávání zcela nezávislý na použité platformě.
Realizovat takový modul jde pak dvěma způsoby - buďto jako samostatná serverová služba nebo jako modul bez využití konstruktů použité platformy, v našem případě React.
Jelikož lze získávat sémantická data skrze \glref{SPARQL} endpoint, který lze dotazovat volat přímo z prohlížeče, je v našem případě vhodnější jít cestou modulu, který bude využíván přímo na frontend.

Tento modul bude pojmenován {\em sfs-api} a bude koncipován tak, aby to bylo samostatná knihovna, kterou lze využít v jakémkoliv Javascript frameworku či prostředí.
Jelikož však chceme implementovat vyhledávač primárně na platformě React, vytvoříme ještě jeden balíček\fnote{Balíčkem je myšlen Node.js balíček definovaný package.json.} jménem {\em react-sfs}, která bude ještě zjednodušovat rozhraní {\em sfs-api} pro React aplikace.
Výsledkem tak budou 3 projekty, balíčky {\em sfs-api} a {\em react-sfs} a finální projekt {\em sfs-react-demo}, který bude naším sémantickým facetovým vyhledávačem a zároveň také ukázkovým příkladem použití těchto balíčků.
Balíčky {\em sfs-api} a {\em react-sfs} budou veřejně publikovány v repozitáři {\em npm}\urlnote{https://www.npmjs.com}, který je primárním repozitářem Node.js balíčků.
Jejich názvy jsou odvozené z SFS\fnote{Zkratka pro \uv{semantic faceted search}, česky sémantické facetové vyhledávání.}, což je název naší knihovny a budeme na ni tak v práci dále odkazovat.

\midinsert
\picw=15cm \cinspic images/sfs-component-diagram.png
\caption/f Diagram komponent sémantického facetového vyhledávače
\endinsert

Součástí publikovaných balíčku nejsou žádné komponenty či webové prvky, které by definovaly vizuální podobu samotných facetů či vyhledávání. 
Záměrně tuto část necháváme na uživateli knihovny, abychom ho nijak neomezovali a nevnucovali mu vizuální podobu.
Může se však inspirovat komponenty v našem vyhledávači {\em sfs-react-demo}.
Díky tomu se dá knihovna snadně integrovat do různých projektů.

\sec sfs-api
V této sekci si popíšeme důležité části balíčku {\em sfs-api}.
Tyto části jsou jádrem celé knihovny SFS.

\secc [facet] Facet
Stav facetu lze definovat jako jeho aktuální možnosti a vybraná hodnota.
Tento stav pro nás bude představovat abstraktní třída {\ssr Facet}.
Jednotlivé typy facetů budou pak podtřídami třídy {\ssr Facet}.
Tento design bude uživateli knihovny SFS umožňovat namodelovat si vlastní facet vytvořením vlastní podtřídy třídy {\ssr Facet}.
To může být užitečné pro specifické typy facetů, které nepokryje knihovna SFS.

Typy facetu se liší hlavně tím, jak se skládají jejich dotazy k získání možností či to, jaké constraints utvářejí.
Při vytváření vlastního facetu bude mít uživatel možnost napsání vlastní logiky pro tyto části.

\begtt
interface FacetConfig {
  id: string,
  predicate: string,
  labelPredicates?: string[],
}
\endtt
Konfigurace třídy {\ssr Facet} je reprezentována rozhraním {\ssr FacetConfig}.

Součástí {\ssr FacetConfig} je {\ssr id}, které je primárním identifikátorem facetu a musí být unikátní v rámci jednoho {\ssr SfsApi}.

{\ssr predicate} je predikát v rámci \glref{RDF} trojice použitých k dotazování možností facetu a následného filtrování dle ní.
Toto může být \glref{IRI} a můžeme využít i klauzule {\ssr PREFIX} definované v {\ssr SfsApiConfig}.

{\ssr labelPredicates} jsou pak predikáty, které chceme použít k najití názvů jednotlivých možností facetu.
Jsou aplikovány v pořadí elementů v poli a pokud žádný nenajde název nebo {\ssr labelPredicates} nedefinujeme bude použito {\ssr id} možnosti, tedy celé její \glref{IRI}.

\secc SfsApi
Rozhraním celého facetového vyhledávání bude třída {\ssr SfsApi}.
Ta bude mít přístup ke všem facetům a zároveň bude konfigurovat facetové vyhledávání jako celek skrze rozhraní níže.
\begtt
interface SfsApiConfig {
  endpointUrl: string,
  baseQuery: string,
  facets: Facet[],
  language: string,
  prefixes?: Prefixes,
}
\endtt

Důležitým atributem je {\ssr baseQuery}, který by měl obsahovat \glref{SPARQL} query, která se používá k dotazování výsledků.
Tato query je klíčová, protože se z ní skládají i query pro možnosti facetů.

Součástí této query musí být proměnné {\ssr _id} a {\ssr _label}.
Proměnná {\ssr _id} představuje primární identifikátor jednotlivých výsledků a {\ssr _label} pak jejich jméno, kterým by měly být označeny.
Na základě tohoto jména jsou pak i fitrovány výsledky v případě fulltextového vyhledávání.
Všechny názvy interních \glref{SPARQL} proměnných jsou prefixovány podtržítkem, tudíž by tak uživatel neměl nazývat jiné své proměnné.

Skrze {\ssr SfsApi} také můžeme iniciovat nové vyhledávání.
To vyresetuje stav všech facetů.
V rámci hledání také můžeme poskytnout {\ssr searchPattern}, což je fráze, kterou by měla obsahovat proměnná {\ssr _label}.
Tato funkce představuje fulltextové vyhledávání zmíněné v sekci popisu facetového vyhledávání \ref[popis-facetoveho-vyhledavani].

\secc Eventy
Hlavní rozhraní pro komunikaci události, které nastanou v rámci facetového vyhledávání je třída {\ssr EventStream}.
Ta využívá komunikační model Publish-Subscribe.
V tomto modelu mohou vydavatelé (publisher) skrze zprostředkovatele (broker) posílat data na nějaké téma (topic).
Odběratelé (subscriber) mohou toto téma odebírat a zprostředkovatel jim všechny data na to téma rozesílá.

\medskip
\picw=15cm \cinspic images/EventStream-diagram.png
\caption/f Diagram znázorňující komunikační model Publish-Subscribe třídy {\ssr EventStream}\fnote{Inspirováno obrázkem z prezentace Stanislava Vítka\cite[nsi-publish-subscribe].}
\medskip

V případě třídy {\ssr EventStream} je ona tím zprostředkovatelem.
Vydavatelé a odběratelé jsou ostatní třídy {\ssr Facet}, {\ssr SfsApi} nebo cokoliv dalšího, co uživatel knihovny SFS přihlásí k odběru (např. komponenty facetů).
Přihlašování probíhá skrze metodu {\ssr on(eventType, callback)}. 
Parametr {\ssr eventType} je typem eventu, který chceme odebírat, {\ssr callback} pak funkce, která se vykoná když event nastane.
\begtt
type EventType =
  | "RESET_STATE"
  | "NEW_SEARCH"
  | "FACET_VALUE_CHANGED"
  | "FETCH_FACET_OPTIONS_PENDING"
  | "FETCH_FACET_OPTIONS_SUCCESS"
  | "FETCH_FACET_OPTIONS_ERROR"
  | "FETCH_RESULTS_PENDING"
  | "FETCH_RESULTS_SUCCESS"
  | "FETCH_RESULTS_ERROR";
\endtt
Eventy v sobě mají další informace podle typu eventu, například {\ssr FetchFacetOptionsSuccessEvent} níže.
\begtt
interface FetchFacetOptionsSuccessEvent {
  type: "FETCH_FACET_OPTIONS_SUCCESS",
  facetId: string,
  options: FacetOption[],
}
\endtt

Díky odebíráním těchto eventů může uživatelské rozhraní na stav facetů.
Pro platformu React toto ještě zjednodušuje balíček {\em react-sfs}, který bude popsán v následují sekci \ref[navrh-react-sfs].


\sec [navrh-react-sfs] react-sfs
Balíček {\em react-sfs} má zjednodušovat rozhraní pro použití knihovny SFS na platformě React.
Poskytuje dvě funkce {\ssr useFacet} a {\ssr useFacetSearch}. 
Tento typ funkcí se ve světě platformy React nazývá \uv{hook}\cite[react-hooks].
Funkce hook umožňují jednoduše sdílet části kódu v různých React komponentách.

Hook {\ssr useFacet} má poskytovat rozhraní pro komponenty jednotlivých facetů.
Poskytuje možnosti facetů, aktuální hodnotu nebo stav požadavku na nové možnosti.
Toto rozhraní je generické, jelikož hodnoty různých druhů facetů mohou mít různé typy.
\begtt
type UseFacet<Value> = (facet: Facet<Value>) => UseFacetResult<Value>;

interface UseFacetResult<Value> {
  options: FacetOption[],
  value: Value | undefined,
  onValueChange: (newValue: Value) => void,
  isFetching: boolean,
  error: any,
}
\endtt

{\ssr useFacetSearch} pak poskytuje podobné rozhraní pro výsledky facetového vyhledávání.
Parametrem je tedy tedy třída {\ssr SfsApi}.
\begtt
type UseFacetSearch = (sfsApi: SfsApi) => UseFacetSearchResult;

interface UseFacetSearchResult {
  results: Results | undefined,
  isFetching: boolean,
  lastSearchPattern: string,
  error: any,
}
\endtt

Tyto rozhraní bylo inspirováno populárními knihovnami jako RTK Query\cite[rtk-query] nebo SWR\cite[swr].

\sec Přístup k facetovému vyhledávání
V sekci \ref[analyza-pristupu-k-facetovemu-vyhledavani] jsme si zanalyzovali dva existující přístupy k facetovému vyhledávání.
Náš přístup bude velmi podobný tomu druhému, tedy SPARQL Faceter\fnote{To nemusí být úplně překvapivé, jelikož je to také knihovna na sémantické facetové vyhledávání.}.
Každý facet (v našem případě už tím můžeme myslet naší třídu {\ssr Facet}) si bude sám sestrojovat dle aktuálního stavu \glref{SPARQL} query na získání jeho možností.
Odesílat tento dotaz na \glref{SPARQL} endpoint však bude vždy třída {\ssr SfsApi}.
Ta doplní query facetu o constraints ostatních facetů, či omezení fulltextovým vyhledáváním, které je také součástí facetového vyhledávání.
Přidá také klauzule {\ssr PREFIX}, které jsou definovány atributem prefixes.

\midinsert
\picw=16cm \cinspic images/sfs-facet-change-sequence-diagram.png
\caption/f Sekvenční diagram změny hodnoty facetu
\endinsert

\midinsert
\picw=16cm \cinspic images/sfs-new-results-sequence-diagram.png
\caption/f Sekvenční diagram získávání nových výsledků
\endinsert
