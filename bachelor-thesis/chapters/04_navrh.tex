\chap[navrh] Návrh
\rfc{dodelat}
Výsledkem této práce je sémantický facetový vyhledávač na platformě React.
Tato kapitola se bude zabývat jeho návrhem.
Nejprve navrhneme jak rozdělit moduly vyhledávače, poté popíšeme architekturu jednotlivých modulů a komunikaci mezi jednotlivými facety.


\sec Moduly
Při návrhu je nutné dbát na to, aby byl vyhledávač rozdělený na moduly vyhledávání a jeho vizualizace.
Toto rozdělení nám umožní, aby byl modul vyhledávání zcela nezávislý na použité platformě.
Realizovat takový modul jde pak dvěmy způsoby - buďto jako samostatná serverová služba nebo jako modul bez využití konstruktů použité platformy, v našem případě React.
Jelikož lze získávat sémantická data skrze \glref{SPARQL} endpoint, který lze lehko volat přímo z prohlížeče, dává v našem případě větší smysl jít cestou modulu, který bude využíván přímo na frontendu.

Tento modul bude pojmenován {\em sfs-api} a bude koncipován tak, aby to bylo samostatná knihovna, kterou lze využít v jakémkoliv Javascript frameworku či prostředí.
Jelikož však chceme implementovat vyhledávač primárně na platformě React, vytvoříme ještě jednu knihovnu jménem {\em react-sfs}, která bude usnadňovat použití {\em sfs-api} v React aplikacích.
Výsledkem tak budou 3 projekty, knihovny {\em sfs-api} a {\em react-sfs} a finální projekt {\em sfs-react-demo}, který bude finálním sémantickým facetovým vyhledávačem a zároveň tak ukázkovým příkladem použití těchto knihoven.
Knihovny {\em sfs-api} a {\em react-sfs} budou publikovány v repozitáři NPM\fnote{https://www.npmjs.com}, který je primárním repozitářem Node.js balíčků.

\medskip
\picw=20cm \cinspic images/sfs-component-diagram.png
\caption/f Diagram komponent sémantického facetového vyhledávače
\medskip


Součástí publikovaných knihoven nejsou žádné komponenty či web elementy, které by přímo definovaly vizuální podobu samotných facetů či vyhledávání. 
Záměrně tuto část necháváme na uživateli, abychom ho nijak neomezovali a nevnucovali mu vizuální podobu, která se mu nemusí líbit.
Obě knihovny však poskytují rozhraní pro tyto web elementy, lze tedy využít i samostatně {\em sfs-api} pro jinou platformu než React.


\sec Architektura

\secc [constraints] Constraints
Při návrhu bylo nutné vyřešit jakým způsobem omezíme možnosti facetů dle jiných aktivních.
V návrhu je toto řešeno pomocí \uv{contraints}.
Constraints nazýváme v této práci (a i samotných knihovnách) \glref{SPARQL} patterny, které omezí výsledky dle aktivních facetů.
Tento nápad a termín byly inspirovány existující knihovnou {\em angular-semantic-faceted-search}\cite[angular-semantic-faceted-search], která je také implementací sémantického facetového vyhledávání.\rfc{možná o ní budu psát už někde předtím}

Abstraktní třída {\ssr Facet} představuje stav facetu. 
V době psaní této práce nabízí knihovna {\em sfs-api} dva typy facetů (dle naší definice z \ref[typy-facetu]) - checkbox facet a select facet.
Uživatel může z této třídy dědit a naimplementovat si vlastní facet.
Toho lze docílit implementováním abstraktních metod {\ssr buildOptionsQuery()}, která by měla vracet query k dotazování možností facetů a {\ssr getFacetConstraints()}, která by měla vracet constraints dle [constraints].



\sec Popis eventů
Facety i API komunikují svůj stav mezi sebou pomocí eventů. 
Pomocí těchto eventů můžeme komunikovat stav i se samotným uživatelským prostředím.
Jsou tedy klíčové k integraci knihovny do jakékoliv aplikace.
Abychom si to lépe přiblížili, ukážeme si proces změny hodnoty facetu a reakci ostatních facetů na sekvenčním diagramu níže.
\medskip
\picw=20cm \cinspic images/sfs-facet-change-sequence-diagram.png
\caption/f Sekvenční diagram změny hodnoty facetu
\medskip


\sec Konfigurace (tohle spíš do implementace asi)
\rfc{pridat listing label}
\begtt 
export const sfsApi = new SfsApi({
  endpointUrl: "https://xn--slovnk-7va.gov.cz/sparql",
  facets: [glosaryFacet, subClassOfFacet],
  queryTemplate:
    `SELECT DISTINCT ?_id ?_label 
WHERE 
  { ?_id a <http://www.w3.org/2004/02/skos/core#Concept> .
    FILTER isIRI(?_id)
    OPTIONAL
      { ?_id rdfs:label ?rdfsLabel 
        FILTER langMatches(lang(?rdfsLabel), "${language}")
      }
    OPTIONAL
      { ?_id skos:prefLabel ?prefLabel 
        FILTER langMatches(lang(?prefLabel), "${language}")
      }
      BIND(coalesce(?rdfsLabel, ?prefLabel, ?_id) AS ?_label) 
  }
ORDER BY ASC(?_label)`,
  prefixes: {
    rdfs: "http://www.w3.org/2000/01/rdf-schema#",
    skos: "http://www.w3.org/2004/02/skos/core#",
    dct: "http://purl.org/dc/terms/",
    dbp: "http://dbpedia.org/property/",
    dbo: "http://dbpedia.org/ontology/",
  },
  language,
});
\endtt

Facet je reprezentován rozhraním {\em Facet} a má své unikátní facetId, dle kterého je dále identifikován.
Jednotlivé typy facetů pak toto rozhraní rozšiřují a implementují jeho metodu \uv{generateSparql}, která se volá při sestavení výsledného \glref{SPARQL} dotazu.
Vracet by měla část \glref{SPARQL} týkající se tohoto facetu.
Podpora dalších typů je plánovaná s dalším vývojem vyhledávače. 
Tento design umožňuje uživateli vytvořit si vlastní typ facetu (implementováním interfacu Facet) pro případy nepokryté knihovnou, s tím, že ho ale stále může kombinovat s ostatními facety.

K připojení jednotlivých facetů používáme modul {\em react-sfs}. 
Ten za pomoci implementovaného React hooku spojí prezentační vrstvu se stavem facetu a je tak bodem komunikace s druhým modulem.
K tomu se používá hlavně subscriber pattern, kde je identifikátorem právě ono facetId.
