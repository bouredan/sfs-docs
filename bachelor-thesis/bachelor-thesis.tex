\input ctustyle3
\input chapters/slovnicek


\worktype [B/CZ]
\faculty {F3}
\department {Katedra počítačů}

\title {Sémantické facetové vyhledávání na platformě React}

\author {Daniel Bourek}
\supervisor {Ing. Martin Ledvinka, Ph.D.}
\date {Květen 2022}

\studyinfo {Softwarové inženýrství a technologie}

\specification {%
\vbox to0pt{\vskip-25mm\centerline{\inspic zadani_bakalarske_prace.pdf }\vss}
}

\abstractCZ {
Práce se zabývá problematiku facetového vyhledávání se zaměřením na jeho použití pro sémantická data.
Zprvu čtenáře seznámí s pojmem sémantický web a klíčovými technologiemi, které se ho týkají. 
Poté zanalyzuje přístupy k facetovému vyhledávání a navrhne vhodný pro sémantická data.
Jádrem práce je návrh a implementace sémantického facetového vyhledávače.
Ten je implementován ve frameworku React, ale je koncipován tak, aby byl použitelný v jakémkoliv JavaScript frameworku.
Celý proces je řádně zdokumentován.
Na závěr je vyhledávač vyhodnocen a porovnán s již existujícím na adrese \url{https://slovník.gov.cz/prohlížeč}.
}

\abstractEN {
Semantic Web is an extension of the current Web which uses semantic data model to express real-world data in a machine-readable format.
This work deals with designing and implementing faceted search for this semantic data.
First, it researches and presents key semantic web concepts and technologies.
It then analyzes approaches to facet search and suggests a suitable approach for semantic data.
The core of the work is the design and implementation of a semantic faceted search engine.
The search engine is implemented in the React framework but it is designed to be usable in any JavaScript framework.
The whole process is properly documented.
Finally, the implemented faceted search is evaluated and compared with the existing one at \url{https://slovník.gov.cz/prohlížeč}.
}

\keywordsCZ {
sémantický web, \nl facetové vyhledávání, SPARQL, RDF, React
}

\keywordsEN {
Semantic Web, facet search, SPARQL, RDF, React
}

\thanks{
Chtěl bych poděkovat vedoucímu této práci, Ing. Martinovi Ledvinkovi Ph.D., za výborné vedení této práce a veškerou pomoc poskytnutou při její tvorbě.
Opravdu si nedokážu představit lepšího vedoucího bakalářské práce.

Dále bych chtěl poděkovat své rodině za neustálou podporu a hlavně trpělivost při mých studiích.
V neposlední řadě také své přítelkyni Cecílii, která mě po celou dobu tvorby práce motivovala, podporovala a podílela se na její jazykové korektuře.
}

\declaration {
Čestně prohlašuji, že jsem předloženou práci vypracoval samostatně a že jsem uvedl veškeré použité informační zdroje 
v souladu s Metodickým pokynem o dodržování etických principů při přípravě vysokoškolských závěrečných prací.\nl\nl
Daniel Bourek\nl
V Praze, 19. 5. 2022
}

\makefront

\input chapters/01_uvod
\input chapters/02_semanticky_web
\input chapters/03_facetove_vyhledavani
\input chapters/04_navrh
\input chapters/05_implementace
\input chapters/06_vyhodnoceni
\input chapters/07_zaver

\input chapters/prilohy

\bye