\input ctustyle3
\input chapters/slovnicek

\worktype [B/CZ]
\faculty {F3}
\department {Katedra počítačů}

\title {Sémantické facetové vyhledávání na platformě React}

\author {Daniel Bourek}
\supervisor {Ing. Martin Ledvinka}
\date {Květen 2022}

\studyinfo {Softwarové inženýrství a technologie}

\specification {%
\vbox to0pt{\vskip-25mm\centerline{\inspic zadani_bakalarske_prace.pdf }\vss}
}

\abstractCZ {
\rfc{předělat abstrakty}
Práce se zabývá problematiku facetového vyhledávání se zaměřením na jeho použití pro sémantická data.
Zprvu čtenáře seznámi s pojmy a technologiemi sémantického webu. 
Pak zanalyzuje přístupy k facetovému vyhledávání a navrhne vhodný pro sémantická data.
Součástí práce je pak návrh modulu sémantického facetového vyhledávače na platformě React s důrazem na jeho přepoužitelnost.
Výsledkem je pak jeho implementace, která je také okomentovaná 
}

\abstractEN {
The thesis deals with the issue of facet searching with regard to semantic data.
First, it introduces the reader to the concepts and technologies of the Semantic Web.
It then analyzes approaches to facet search and suggests suitable one for semantic data.
Core of the work is the development of a semantic facet search prototype using Javascript framework React and emphasising on the reusabilty of this module.
Thesis is successful in fulfilling its goals.
}

\thanks{
\rfc{předělat poděkování}
Rád bych poděkoval vedoucímu této práce Ing. Martinovi Ledvinkovi za výborné vedení, mnoho doporučení a velkou ochotu na konzultacích při vypracovávání této práce.
Dále bych chtěl poděkovat své rodině za jejich neustálou podporu při mých studiích.
}

\declaration {
Prohlašuji, že jsem předloženou práci vypracoval samostatně.

Daniel Bourek
V Praze, 15. 5. 2022
}

\draft
\makefront

\input chapters/01_uvod
\input chapters/02_semanticky_web
\input chapters/03_facetove_vyhledavani
\input chapters/04_navrh
\input chapters/05_implementace
\input chapters/06_zaver

\bibchap
\usebib/c (iso690) chapters/literatura

\app Slovníček 
\makeglos

\bye