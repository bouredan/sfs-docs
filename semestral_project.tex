\input ctustyle3

\worktype [O/CZ]
\faculty {F3}
\department {Katedra počítačů}

\title {Sémantické facetové vyhledávání na platformě React}
\subtitle {subtitle}

\author {Daniel Bourek}
\date {Září 2021}

\studyinfo {Softwarové inženýrství a technologie}

\workname {Semestrální projekt}


\abstractEN {testing}
\abstractCZ {testovani}
\declaration {prohlaseni}
\makefront

\chap Úvod

\chap Úvod do sémantických technologií
	\sec Historie, Web verze,
	https://cambridgesemantics.com/blog/semantic-university/intro-semantic-web/towards-the-semantic-web/ podobný vysvětlení jako ve videu

	\sec Vývoj v posledních letech
	tady dát jak je to jen 10 let starý vlastně

	velký pokrok v RDF, OWL a SPARQL

	odkazy k částem dokumentů + google vyhledávání taky

\chap Facetové vyhledávání
	\sec Popis
	\sec Typy facetů
		\secc Basic facet
		Základní typ facetu s vybíráním dle hodnoty kategorie.

		\secc Range facet
		Facet s možností nastavení rozsahu.

		\secc Bucket facet
		Podobné jako basic facet, akorát kategorie jsou vytvořeny dle částí rozsahu.
		
	\sec Srovnání přístupů (platforem)

\chap Slovník.gov.cz
	\sec Popis

	\sec Popis endpointu
	http://vos.openlinksw.com/owiki/wiki/VOS/VOSSparqlProtocol

\chap Popis implementace
https://www.npmjs.com/package/@tpluscode/sparql-builder

\bye